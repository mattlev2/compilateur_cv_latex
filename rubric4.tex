%%% rubric.tex --- Example of using CurVe.

%% Copyright (C) 2000, 2001, 2002, 2003, 2004, 2005, 2010 Didier Verna.

%% Author:        Didier Verna <didier@lrde.epita.fr>
%% Maintainer:    Didier Verna <didier@lrde.epita.fr>
%% Created:       Thu Dec 10 16:04:01 2000
%% Last Revision: Mon Dec  6 11:04:22 2010

%% This file is part of CurVe.

%% CurVe may be distributed and/or modified under the
%% conditions of the LaTeX Project Public License, either version 1.1
%% of this license or (at your option) any later version.
%% The latest version of this license is in
%% http://www.latex-project.org/lppl.txt
%% and version 1.1 or later is part of all distributions of LaTeX
%% version 1999/06/01 or later.

%% CurVe consists of the files listed in the file `README'.


%%% Commentary:

%% Contents management by FCM version 0.1.


%%% Code:


\begin{rubric}{Publications et communications}
                    \subrubric{Communications}
                    
                
                    
                    \entry*
                \enquote{L'évolution du \textit{Regimiento de los prínçipes}
                    (1345-1494), au service du pouvoir ?}Séminaire international:\href{https://recherche.univ-pau.fr/fr/accueil/cpim.html}{ Les cultures
                    politiques dans la Péninsule ibérique et au Maghreb}, Bielle,
                    octobre 2019. 
                    
                    \entry*
                \enquote{Construire l'édition numérique d'une tradition textuelle complexe:
                    propositions pour le \textit{Regimiento de los prínçipes}}. Atelier
                    IRPALL \enquote{La fabrique du texte}, Toulouse, février
                    2019.
                \subrubric{Publications}
            
                    
                    \entry*
                \textit{Modèle de reconnaissance optique de caractères ocropy
                    - Incunables sévillans
                    1494-1500}.
                    22 décembre 2018. \href{https://doi.org/10.5281/zenodo.2504783}{\includegraphics[scale=0.55]{../img/zenodo2504784.png}} \end{rubric}

%%% rubric.tex ends here

%%% Local Variables:
%%% mode: latex
%%% TeX-master: "raw"
%%% End: