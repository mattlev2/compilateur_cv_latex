%%% rubric.tex --- Example of using CurVe.

%% Copyright (C) 2000, 2001, 2002, 2003, 2004, 2005, 2010 Didier Verna.

%% Author:        Didier Verna <didier@lrde.epita.fr>
%% Maintainer:    Didier Verna <didier@lrde.epita.fr>
%% Created:       Thu Dec 10 16:04:01 2000
%% Last Revision: Mon Dec  6 11:04:22 2010

%% This file is part of CurVe.

%% CurVe may be distributed and/or modified under the
%% conditions of the LaTeX Project Public License, either version 1.1
%% of this license or (at your option) any later version.
%% The latest version of this license is in
%% http://www.latex-project.org/lppl.txt
%% and version 1.1 or later is part of all distributions of LaTeX
%% version 1999/06/01 or later.

%% CurVe consists of the files listed in the file `README'.


%%% Commentary:

%% Contents management by FCM version 0.1.


%%% Code:
 \begin{rubric}{Cursus}
                                \entry*[2018-2019]
                            Première année de doctorat sous la direction de Carlos \textsc{Heusch}\href{https://www.idref.fr/055838413}{\textsuperscript{\includegraphics[scale=0.025]{img/idref.png}}}. Laboratoire de
                        rattachement: \href{http://ciham.ish-lyon.cnrs.fr/}{CIHAM} (UMR
                        6848). Démarches en cours pour monter une co-direction avec Jesús \textsc{Rodríguez Velasco}\href{https://orcid.org/0000-0002-3848-9230}{\includegraphics[scale=0.025]{img/orcid.png}} et
                        Marjorie \textsc{Burghart}\href{https://www.idref.fr/178995819}{\textsuperscript{\includegraphics[scale=0.025]{img/idref.png}}}.
                    
                                \entry*[2017-2018]
                            Préparation de la thèse (4ème année à l'ÉNS de Lyon). 
                    
                                \entry*
                            Suivi d'un atelier en édition numérique à Cambridge (Royaume Uni) en
                        avril 2018, dans la continuité du DEMM (\textit{Digital Encoding of Medieval
                        Manuscripts}).
                    
                                \entry*
                            Participation à la session 2018 de la formation EDEEN (École d'été en
                        humanités numériques) à Grenoble, organisée par l'Université Grenoble-Alpes
                        et la MSH Alpes (Grenoble, 28 mai-2 juin).
                    
                                \entry*
                            Participation à l'atelier 2018 du consortium CAHIER (Montpellier,
                        26-29 juin 2018).
                    
                                \entry*[2016-2017]
                            Préparation et obtention du concours de l'Agrégation d'Espagnol. 
                    
                                \entry*[2015-2016]
                             Master II d'études hispanophones du département d'études lusophones
                        et hispanophones de l'ÉNS de Lyon. Suivi des cours Master \textit{Patrimonio
                        Textual y humanidades digitales} de l’Université de Salamanque dans
                        le cadre du programme Erasmus. Mémoire préparé sous la direction de Carlos \textsc{Heusch}\href{https://www.idref.fr/055838413}{\textsuperscript{\includegraphics[scale=0.025]{img/idref.png}}} et de Francisco \textsc{Bautista}\href{https://orcid.org/0000-0002-2676-0388}{\includegraphics[scale=0.025]{img/orcid.png}}: \enquote{La
                        version B du \enquote{\textit{Regimiento de los príncipes}
                        glosado}}. Note obtenue: 19. 
                    
                                \entry*
                            Suivi de la formation \textit{Digital Encoding of Medieval
                        Manuscripts} (DEMM) financée par l’Union Européenne. Trois semaines de
                        stage à Vienne (Abbaye de Klosterneuburg), Lyon (ÉNS de Lyon), Londres
                        (Queen Mary University). 
                    
                                \entry*[2014-2015]
                            Master I d'études hispanophones au sein du département d'études
                        lusophones et hispanophones de l'ÉNS de Lyon. Spécialisation en philologie
                        castillane et en étude de textes du bas-Moyen Âge. Mémoire préparé sous la
                        direction de Carlos Heusch:\textit{Edición crítica de dos textos de Pedro de
                        Chinchilla}. Note obtenue: 18.
                    
                                \entry*
                            Stage d’ecdotique des Sources chrétiennes (Lyon, février
                        2015).
                    
                                \entry*[2013-2014]
                            Séminaire d’édition de textes de Gisèle
                        \textsc{Besson}, ÉNS Lyon.
                    
                                \entry*
                            Stage d'introduction au Grec ancien (ILOAM, Université de Lausanne,
                        été 2014).
                    
                                \entry*[2012-2013]
                            Khâgne, Lycée Joffre (Montpellier). Préparation aux concours des ÉNS.
                        Admis à l'ÉNS de Lyon; admissible à l'ÉNS de Paris.
                    
                                \entry*[2011-2012]
                            Hypokhâgne, Lycée Joffre (Montpellier).
                    \end{rubric}




%%% rubric.tex ends here

%%% Local Variables:
%%% mode: latex
%%% TeX-master: "raw"
%%% End: