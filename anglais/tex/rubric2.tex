%%% rubric.tex --- Example of using CurVe.

%% Copyright (C) 2000, 2001, 2002, 2003, 2004, 2005, 2010 Didier Verna.

%% Author:        Didier Verna <didier@lrde.epita.fr>
%% Maintainer:    Didier Verna <didier@lrde.epita.fr>
%% Created:       Thu Dec 10 16:04:01 2000
%% Last Revision: Mon Dec  6 11:04:22 2010

%% This file is part of CurVe.

%% CurVe may be distributed and/or modified under the
%% conditions of the LaTeX Project Public License, either version 1.1
%% of this license or (at your option) any later version.
%% The latest version of this license is in
%% http://www.latex-project.org/lppl.txt
%% and version 1.1 or later is part of all distributions of LaTeX
%% version 1999/06/01 or later.

%% CurVe consists of the files listed in the file `README'.


%%% Commentary:

%% Contents management by FCM version 0.1.


%%% Code:

\begin{rubric}{Teaching}
                                \entry*[2018-2019]
                                
                            Organization and animation of a training course focused on XSLT (\href{https://cosme.hypotheses.org/1117}{COSME XSLT Course 2019}).
                        Link to git repository: \href{https://github.com/gabays/Cours\_COSME\_2019}{here}.
                    
                                \entry*
                            Graduate TD at ÉNS de Lyon, \enquote{\textit{Version classique}}
                        (translation of texts of the spanish Siglo de Oro)
                    
                                \entry*
                            Graduate and undergraduate TD, \enquote{\textit{Méthodologie de la
                        recherche}}. I focused my course on the economic, juridic,
                        technical issues of the \enquote{scientific publication}, on one hand; on
                        the other hand I did an introduction to LaTeX for researchers.
                    
                                \entry*[2017-2018]
                                
                            Preparation sessions for oral exams in Spanish in CPGE (\textit{Classe
                        Préparatoire aux Grandes Écoles}) at Lycée La Martinière Duchère and
                        Lycée Récamier, Lyon.
                    \end{rubric}

%%% rubric.tex ends here

%%% Local Variables:
%%% mode: latex
%%% TeX-master: "raw"
%%% End: